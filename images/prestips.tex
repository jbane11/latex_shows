%\documentclass[11pt]{beamer}
\documentclass[dvips,11pt]{beamer}
%\documentclass[handout,dvips,11pt,gray]{beamer}

% Setup appearance:

%\usetheme{Darmstadt}
%\usetheme{Frankfurt}
%\usetheme{Warsaw}
\usetheme[secheader]{Boadilla}
%\setbeameroption{hide notes}
%\setbeameroption{show notes}
%\setbeameroption{show only notes}
\usefonttheme[onlylarge]{structurebold}
\setbeamerfont*{frametitle}{size=\normalsize,series=\bfseries}
\setbeamertemplate{navigation symbols}{}
\setbeamertemplate{note page}[plain]

% Standard packages

\usepackage[english]{babel}
\usepackage[latin1]{inputenc}
\usepackage{times}
\usepackage[T1]{fontenc}
\usepackage{pst-all}              % PStricks
\usepackage{graphicx}             % Graphics formats
%\usepackage{multimedia}

%\usepackage{pgfpages}
%\pgfpagesuselayout{4 on 1}[letterpaper,landscape,border shrink=5mm]

%\logo{\includegraphics[width=0.05\textwidth]{utlogo.jpg}}
\logo{\includegraphics[width=0.05\textwidth]{utlogo.eps}}

% Author, Title, etc.

\title[\LaTeX\ Tips and Tricks] 
{%
  Presentations with \LaTeX: Tips and Tricks%
}

\author[Saum]
{
  Michael A. Saum 
}

\institute[Dept of Mathematics, UTK]
{
  Department of Mathematics \\
  The University of Tennessee, Knoxville
}

\date[10/10/07]
{October 10, 2007}

\begin{document}
%\bibliographystyle{apalike}
%\bibliographystyle{unsrt}

\begin{frame}
  \titlepage
\note<1>{Always have a title page!}
\end{frame}

\begin{frame}{Outline}
  \tableofcontents[hideallsubsections]
\note<1>{Table of contents really only necessary when talk is 45 min or more.}
\end{frame}


\section{Introduction}

\subsection{Presentations: Good and Bad}

\begin{frame}{Bad Presentations}
What makes a \alert{Bad} Presentation?
\begin{itemize}
\item Not targeted to audience.
\item Not well thought out. 
\item Time issues.
\item Too much on each slide.
\item Too many slides.
\end{itemize}
\note<1>{Itemize environments work well.}
\end{frame}

\begin{frame}{Good Presentations}
What makes a \alert{Good} Presentation?
\begin{enumerate}
\item Targeted to audience, keeps audience engaged.
\item Well thought out.
\item Conveys information.
\item Right on Time.
\end{enumerate}
\note<1>{Enumerate environments also work well.}
\end{frame}

\section{Slides with \LaTeX}

\subsection*{}

\begin{frame}{Why \LaTeX ?}
\begin{itemize}
\item Mathematicians use \LaTeX\ for papers.
\item \LaTeX\ typesets mathematics much better than Powerpoint or Word.
\item The \LaTeX\ system is \alert{free}!
\item Worry about the content, not font size, placement, etc. 
\item The \LaTeX\ system is \alert{FREE}!!
\item The end result is an Adobe PDF file, which is extremely portable and can be
displayed on almost any platform.
\end{itemize}
\note<1>{\LaTeX\ is not really a system, but a set of macros built on top of the \TeX\ language.}
\end{frame}

\begin{frame}{Choices to Make}
\begin{itemize}[<+->]
\item Linux or Windows?
\item User Interface - Text editor?
\item Output Choices - Presentation or Handout?
\end{itemize}
\note<1>[item]{Note incremental display of lines.}
\note<1>[item]{On linux systems, everything \LaTeX\ related is
usually contained in {\tt /usr/share/texmf/} subdirectory (including documentation) and is called the
{\tt tetex} distribution.  On Windows systems, one should install the MikTeX system and ensure it is
the latest version.  Note that the MikTeX system is basically the same {\em source code} as that contained
in the the tetex distribution.  This allows for ease of transport of the {\tt .tex} files between systems
and platforms.}
\note<1>[item]{ Choices include {\tt vi}, {\tt emacs}, and {\tt winedt} (windows only).  Note that if working on 
Windows, the winedt ''official'' version costs \$.  However, the trial version works just as well.  Note
also that the winedt/MikTeX system has a nice user interface controlling creation of {\tt .dvi, .ps, and .pdf}
files at the click of a button.  Power users do not use GUI interfaces and therefore use linux.}
\note<1>[item]{ If one desires a handout, one usually will desire multiple slides on a page, or include notes about each
slide.  Usually, one will want handouts to be printed in black and white.  Presentations will usually have
color associated with them.  Switching between these modes can easily be done, as will be shown later.}
\end{frame}

\begin{frame}{\LaTeX\ Presentation Support - History}
\begin{itemize}
\item foiltex
\item seminar
\item prosper
\item beamer
\end{itemize}
\note<1>[item]{{\tt foiltex} was used in the early days of \LaTeX.}
\note<1>[item]{{\tt seminar} is nice and simple, but not much in the way of color enhancement.}
\note<1>[item]{{\tt prosper} was a major step forward in that it grouped together different elements
to present a cohesive ''template'' based on particular ''themes''.  However, it was difficult
to fine tune and to get exactly right.}
\note<1>[item]{{\tt beamer} comes close to providing a complete set of interchangeable elements (themes)
which allow one to focus more on the content rather than the appearance.}
\end{frame}

\section{Using the {\tt beamer} Package}

\begin{frame}[fragile=singleslide]{Framed!}
One should generally use one frame per slide.
\begin{block}{A frame}
\small
\begin{verbatim}
\begin{frame}{frame title}
...  blah ...
\end{frame}
\end{verbatim}
\end{block}
\note<1>[item]{Note the use of a boxed environment.}
\note<1>[item]{Note the use of ''fragile'' in the frame declaration.  This allows use of verbatim
environment.}
\note<1>[item]{Note the change in text size via ''small'' command.  To get really small, use ''tiny'' command.}
\end{frame}

\begin{frame}{More Choices}
\begin{itemize}
\item Presentation Themes
\begin{itemize}
\item With/Without Navigation bars?
\item Tree-like Navigation bar?
\item Table of Contents sidebar? 
\item Mini-Frame Navigation? 
\end{itemize}
\item Outer Themes
\item Inner Themes
\item Color Themes
\item Font Themes
\end{itemize}
\note<1>[item]{Note the nesting of the itemize environments.}
\note<1>[item]{Navigation bars sometimes take away from the presentation.  Use with caution!}
\note<1>[item]{Outer Themes govern how frame title, navigation, etc. appear.}
\note<1>[item]{Inner Themes govern how list environments appear, etc.}
\note<1>[item]{Sometimes choosing different Font and Color Themes don't work well together.}
\end{frame}

\begin{frame}[fragile=singleslide]{Templates, I}
\begin{block}{Setup}
\tiny
\begin{verbatim}
%\documentclass[11pt]{beamer}
\documentclass[dvips,11pt]{beamer}
%\documentclass[handout,dvips,11pt,gray]{beamer}

% Setup appearance:

%\usetheme{Darmstadt}
%\usetheme{Frankfurt}
%\usetheme{Warsaw}
\usetheme[secheader]{Boadilla}
%\setbeameroption{hide notes}
%\setbeameroption{show notes}
%\setbeameroption{show only notes}
\usefonttheme[onlylarge]{structurebold}
\setbeamerfont*{frametitle}{size=\normalsize,series=\bfseries}
\setbeamertemplate{navigation symbols}{}
\setbeamertemplate{note page}[plain]
\end{verbatim}
\end{block}
\note<1>{Change options depending on needs.}
\end{frame}

\begin{frame}[fragile=singleslide]{Templates, II}
\begin{block}{Packages}
\tiny
\begin{verbatim}
% Standard packages

\usepackage[english]{babel}
\usepackage[latin1]{inputenc}
\usepackage{times}
\usepackage[T1]{fontenc}
\usepackage{pst-all}              % PStricks
\usepackage{graphicx}             % Graphics formats
%\usepackage{multimedia}

%\usepackage{pgfpages}
%\pgfpagesuselayout{4 on 1}[letterpaper,landscape,border shrink=5mm]

%\logo{\includegraphics[width=0.05\textwidth]{utlogo.jpg}}
\logo{\includegraphics[width=0.05\textwidth]{utlogo.eps}}

% Author, Title, etc.

\title[\LaTeX\ Tips and Tricks]{Presentations with \LaTeX: Tips and Tricks}

\author[Saum]{ Michael A. Saum}
\institute[Dept of Mathematics, UTK]
{Department of Mathematics \\ The University of Tennessee, Knoxville}

\date[10/10/07]{October 10, 2007}

\begin{document}
... frames ...
\end{document}
\end{verbatim}
\end{block}
\note<1>[item]{This is what I have found works well.  Your mileage may vary.}
\note<1>[item]{A nice set of basic templates is contained in {\tt /usr/share/texmf/tex/latex/beamer/solutions/}
subdirectory.}
\end{frame}

\begin{frame}[fragile=singleslide]{Handouts}
To print out a presentation (with notes) for handouts, I use the following:
\tiny
\begin{verbatim}
\documentclass[handout,dvips,11pt,gray]{beamer}
\usetheme[secheader]{Boadilla}
\setbeameroption{show notes}

\usefonttheme[onlylarge]{structurebold}
\setbeamerfont*{frametitle}{size=\normalsize,series=\bfseries}
\setbeamertemplate{navigation symbols}{}
\setbeamertemplate{note page}[plain]

\usepackage[english]{babel}
\usepackage[latin1]{inputenc}
\usepackage{times}
\usepackage[T1]{fontenc}
\usepackage{pst-all}              % PStricks

\usepackage{pgfpages}
\pgfpagesuselayout{4 on 1}[letterpaper,landscape,border shrink=5mm]
\end{verbatim}
\note<1>{Note that if you are not using PStricks, one can use pdfLaTeX and remove dvips option in documentclass.}
\end{frame}

\begin{frame}{Getting output}
\small
\begin{block}<1->{Windows}
If on windows using the {\tt winedt} text editor, one should press the ''texify'' (or pdf texify) button, depending
on whether one is using postscript or not.
\end{block}
\begin{block}<2->{Linux, postscript}
{\tt latex filename} (twice) \\
{\tt dvips filename -o} \\
{\tt ps2pdf13 filename}
\end{block}
\begin{block}<3->{Linux, pdf}
{\tt pdflatex filename} (twice)
\end{block}
\note<1>{Note the use of incremental display of different blocks.}
\end{frame}

\section{Graphics}

\subsection*{}

\begin{frame}{Type of Graphics}
\begin{itemize}
\item Inserting graphics into a presentation is a \alert{good} idea, i.e., sometimes a picture
is really worth a thousand words.
\item \LaTeX\ has available a number of packages which allow for simple drawings to be done
within the \LaTeX\ document itself.  
\item  \LaTeX\ also has available packages which allow for importing of external document files, such
as {\tt .ps, .eps, .jpg, .gif}, etc.
\item {\tt beamer} allows for inclusion of ''movies'' in a presentation also.
\end{itemize}
\note<1>{An animated {\tt .gif} file is also a movie!}
\end{frame}

\begin{frame}{Guidelines}
\begin{itemize}
\item If one is including only external {\tt .ps} or {\tt .eps} files, then one should not use pdf\LaTeX.
\item To include external graphics files, use the {\tt graphicx} package and scale the included graphic
by some fraction of the ''textwidth'' variable.
\item One should ensure that all external graphics files to be included are of the same type.
\item If one is required to ''draw'' simple diagrams, I recommend using the {\tt PStricks} (postscript) or
the {\tt pgf/TikZ} (non-postscript) package.
\item For graphs or charts, make sure the labels are readable.
\end{itemize}
\note<1>{One can do just about any type of drawing with PStricks, but it does not work with pdf\LaTeX.  
pgf/TikZ is the new kid on the block, and while not as powerful as PStricks, it still can do a lot, especially for
simple diagrams.}
\end{frame}


\begin{frame}{Sample Drawing with PStricks}
\psset{unit=0.1\textwidth}
\begin{center}
\begin{pspicture}(-4.5,-1.5)(4.5,2.5)
\psarcn[linewidth=2pt]{->}(-3,1){1.2}{80}{45}
\psarcn[linewidth=2pt]{->}(3,1){1.2}{80}{45}
\pscircle(-3,-0.5){0.1}
\pscircle(3,-0.5){0.1}
\pscircle(-3,1){1}
\pscircle[linestyle=none,fillstyle=solid,fillcolor=black](-3,1){0.4}
\pscircle(3,1){1}
\pscircle[linestyle=none,fillstyle=solid,fillcolor=black](3,1){0.6}
\pspolygon[linewidth=1.2pt](-2,-1)(2,-1)(2,0)(-2,0)
\psline[linewidth=1.2pt](-3.4,1)(-3.1,-0.5)
\psline[linewidth=1.2pt](-3,-0.6)(3,-0.6)
\psline[linewidth=1.2pt](3.6,1)(3.1,-0.5)
\rput(0,0.3){\small Deposition Zone}
\psline[linewidth=2pt]{<-}(-2.75,-0.35)(-2.2,-0.35)
\rput[t](-2.25,0.1){\small V}
\psline{|<->|}(-2,-1.2)(2,-1.2)
\rput(0,-1.4){\small 2L}
\end{pspicture}
\end{center}
\vspace{-10pt}
\centerline{Simple Reel-to-Reel Processing System (side view)}
\note<1>[item]{Note that there is no need to put the picture inside of a ''figure'' environment.  Figure
environments are useful where one is referencing figures by number.  However, this is hard to do in a 
presentation.}
\note<1>[item]{Note the simulation of a figure title by using the ''centerline'' command.}
\end{frame}

\begin{frame}{Sample Inclusion of External Graphic}
\begin{center}
\includegraphics[width=0.59\textwidth,angle=-90]{1Dkmcbcf.ps}
\end{center}
\note<1>[item]{Scaling width to full textwidth may move the graphic off the page.  This may be an iterative process 
to get it to look right.}
\end{frame}

\begin{frame}{Columns}
\begin{columns}[c]
\column{0.45\textwidth}
\onslide<1-> Exact Solution \\ \onslide<2-> Computed Solution
\column{0.45\textwidth}
\begin{center}
\onslide<1-> \includegraphics[width=0.50\columnwidth,angle=-90]{E4f4_u.eps} \\
\onslide<2-> \includegraphics[width=0.50\columnwidth]{E4f4d4_s.tif.eps}
\end{center}
\end{columns}
\note<1>[item]{Clever use of columns and external graphics.}
\note<1>[item]{Graphics are probably too small though.}
\end{frame}

\begin{frame}{Movies}
\begin{itemize}
\item \href{run:./viewf6d2a_e}{f6d2a est -- UNIX}
\item \href{run:f6d2a_e.gif}{f6d2a est -- Windoze}
\end{itemize}
\note<1>[item]{Note the use of the package {\tt hyperref} which allows for running external programs.  Note also that
{\tt hyperref} is loaded by beamer anyway, so there is no need to explicitly load it.}
\note<1>[item]{{\tt beamer} has other ways of dealing with this which may be better.}
\end{frame}

\section{Summary}

\subsection*{}

\begin{frame}{Comments}
\begin{itemize}
\item There are many different ways of accomplishing the same task with \LaTeX.  Once one finds something that
works, stick with it until one finds a better way of doing it.
\item One should definitely explore different themes with {\tt beamer}.  Make extensive use of the references listed
at the end.
\item Putting too many fancy things into a presentation takes away from the purpose of the presentation, i.e., the 
content.  Sometimes basic is better.
\item With {\tt beamer} flexibility is the key.  Note that {\tt beamer} provides a powerful alternative to 
using Microsoft Powerpoint.
\end{itemize}
\end{frame}

\begin{frame}{References}
\begin{itemize}
\item {\tt beamer} documentation
\begin{itemize}
\item {\tt /usr/share/texmf/tex/latex/beamer/doc}
\item {\tt /usr/share/texmf/tex/latex/beamer/examples}
\item {\tt /usr/share/texmf/tex/latex/beamer/solutions}
\item {\tt /usr/share/texmf/tex/latex/beamer/themes}
\end{itemize}
\item {\tt PStricks}: {\tt http://tug.org/PSTricks/}
\item {\tt pgf/TikZ}: {\tt /usr/share/texmf/tex/latex/pgf/doc/generic/pgf}
\item {\tt \LaTeX\ Navigator}: {\tt http://tex.loria.fr/}
\end{itemize}
\end{frame}

\end{document}
