\documentclass[a4paper]{article}
%\usepackage{simplemargins}

%\usepackage[square]{natbib}
\usepackage{amsmath}
\usepackage{amsfonts}
\usepackage{amssymb}
\usepackage{graphicx}


\begin{document}

\pagenumbering{gobble}

\Large
 \begin{center}
The EMC Effect for the A=3 Mirror Nuclei.\\ 

\hspace{10pt}

% Author names and affiliations
\Large
Jason Bane\\

\hspace{10pt}

\normalsize  
University of Tennessee\\
Thomas Jefferson National Accelerator Facility\\
jbane1@vols.utk.edu\\

\end{center}

\hspace{10pt}

\large
The European Muon Collaboration(EMC) discovered an unexpected and puzzling result in 1983 when comparing the deep inelastic scattering nuclear structure functions of Deuterium and Iron. The per-nucleon structure functions where found to be different for the two nuclei, rather than a simple average over the proton and neutron structure functions.  In subsequent experiments, this phenomenon was confirmed for additional nuclei, with the magnitude approximately scaling with the density.  The exact mechanism leading to this in-medium structure function modification has not been identified.  I will discuss recent results from JLab exploring the EMC effect by using an electron beam to probe two mirror nuclei, Helium-3 and Tritium. 

\end{document}